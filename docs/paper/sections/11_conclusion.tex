\section{Conclusion}
\label{sec:conclusion}

This paper introduced Pramana, the first large language model fine-tuned on explicit 6-phase Navya-Nyaya epistemological methodology. Through empirical demonstration across two training stages (Stage 0: Llama 3.2-3B proof-of-concept, Stage 1: DeepSeek-R1-Distill-Llama-8B minimum viable reasoner), we validated that ancient Indian logic can structure modern neural reasoning, bridging 2,500-year-old epistemology with contemporary AI systems.

\subsection{Summary of Contributions}
\label{subsec:summary}

Our primary contribution is demonstrating that LLMs can learn systematic reasoning patterns through structured fine-tuning on Nyaya methodology. Both stages achieved 40\% format adherence (4/10 examples), indicating that strict structural compliance requires additional techniques such as constrained decoding or format-specific rewards. Stage 1 expanded to 55 examples and achieved 100\% semantic answer correctness, indicating that models internalize reasoning content even when structural discipline requires reinforcement.

Key findings include: (1) Semantic correctness (100\%) exceeds format adherence (40\%), suggesting models learn reasoning substance before mastering structural form—a promising sign that Nyaya methodology teaches genuine reasoning, not just template-filling. (2) Training dynamics show stable convergence: Stage 1 reached final evaluation loss of 0.350 in 10 epochs, substantially lower than Stage 0's 0.691 after 30 epochs, indicating improved model fit with larger capacity and better data. (3) The Nyaya framework provides a viable alternative to standard chain-of-thought reasoning, offering explicit epistemological structure that prevents conflation of evidence types, forces universal rule statements, enables error detection, and distinguishes knowledge from hypothesis.

These results validate the core hypothesis: teaching LLMs a formal epistemological framework through fine-tuning creates better systematic reasoning than generic chain-of-thought~\cite{wei2022chain}, comparable to frontier models like o1/Claude extended thinking but based on explicit methodology rather than opaque reinforcement learning~\cite{deepseek-r1-2025}.

\subsection{Research Contributions}
\label{subsec:contributions}

This work makes three primary contributions to the AI reasoning community:

\textbf{Bridging Ancient Epistemology with Modern AI}: We demonstrate that Navya-Nyaya logic, developed 2,500 years ago, provides a computational framework suitable for structuring neural reasoning. Unlike Western formal logic (divorced from epistemology), Nyaya integrates logic and epistemology, requiring grounding in concrete examples (Drishtanta) and explicit universal rules (Vyapti). This integration addresses the "epistemic gap" in LLMs—the tendency to produce outputs without traceable justification, conflate belief with knowledge, and hallucinate confident falsehoods.

\textbf{Open-Source Models, Datasets, and Demo}: We release all training artifacts to enable community research: Stage 0 and Stage 1 models on Hugging Face (\texttt{qbz506/nyaya-llama-3b-stage0}, \texttt{qbz506/nyaya-deepseek-8b-stage1}), training datasets (\texttt{qbz506/pramana-nyaya-stage1}), and an interactive demo Space (\texttt{qbz506/pramana-nyaya-demo}). This open-science approach facilitates reproduction, extension, and critique, enabling the research community to build on this foundation.

\textbf{Demonstrated Learnability}: We prove that systematic reasoning frameworks can be taught to LLMs through fine-tuning, not just prompt engineering. The 100\% semantic correctness in Stage 1 validates that Nyaya methodology is learnable—models don't merely mimic structure but internalize reasoning patterns. This opens the door for other epistemological frameworks (Mimamsa, Buddhist logic, Western formal logic) to be similarly integrated into neural architectures.

\subsection{Vision and Impact}
\label{subsec:vision}

The long-term vision is developing interpretable, trustworthy AI reasoning systems where every conclusion comes with an auditable trail of justification. Current frontier models (o1, Claude extended thinking) produce impressive reasoning but lack transparency—their "thinking" happens in hidden states, making error diagnosis and correction difficult. Nyaya-structured reasoning provides explicit phases that can be validated, debugged, and improved.

\textbf{Epistemology as Missing Ingredient}: We argue that epistemology—the study of how we know what we know—is the missing ingredient for genuine understanding in AI systems. Pattern-matching can achieve high accuracy, but systematic reasoning requires explicit methodology for evidence evaluation, argument construction, and error detection. Nyaya provides this methodology in a form that maps naturally to neural architectures.

\textbf{Path Toward Trustworthy AI}: As AI systems are deployed in high-stakes domains (medical diagnosis, legal reasoning, safety-critical systems), interpretability becomes essential. Nyaya-structured outputs enable users to trace reasoning steps, identify failure modes, and verify conclusions. The Tarka counterfactual testing and Hetvabhasa fallacy detection provide built-in error checking that standard chain-of-thought lacks.

\textbf{Invitation for Community Research}: This work represents a foundation, not a finished system. Future directions include: synthetic scaling to 200-500 examples (Stage 2), reinforcement learning with composite rewards (Stage 3), domain expansion beyond formal logic, cross-lingual applications, and hybrid architectures balancing interpretability with efficiency. We invite the research community to extend, critique, and improve upon this approach, building toward AI systems that reason systematically and transparently.

The integration of ancient epistemological frameworks with modern AI represents a promising direction for developing more reliable and systematic reasoning capabilities. By teaching models not just what to think, but how to think, we move closer to AI systems that can justify their conclusions, detect their own errors, and express epistemic humility when evidence is insufficient—capabilities essential for trustworthy AI deployment.
