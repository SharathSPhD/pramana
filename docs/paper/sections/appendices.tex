\section{Appendices}

\subsection{Complete Nyaya Glossary}
\label{app:glossary}

Table~\ref{tab:glossary} provides a comprehensive glossary of Nyaya terminology used throughout this work. All terms are Sanskrit philosophical concepts from the Navya-Nyaya tradition, adapted for computational reasoning.

\begin{table}[h]
\centering
\caption{Nyaya terminology glossary.}
\label{tab:glossary}
\begin{tabular}{lp{8cm}}
\toprule
Term & Definition \\
\midrule
Samshaya & Doubt; systematic uncertainty classification. Five types: Samana Dharma Upapatti (multiple entities share properties), Aneka Dharma Upapatti (single entity has multiple conflicting properties), Vipratipatti (contradictory testimony), Upalabdhi Avyavastha (uncertainty about perception validity), Anupalabdhi Avyavastha (uncertainty from absence of evidence). \\
Pramana & Valid means of knowledge. Four types recognized in Nyaya: Pratyaksha (perception), Anumana (inference), Upamana (comparison), Shabda (testimony). \\
Pratyaksha & Direct perception through the senses. In computational context, refers to observable facts directly stated in the problem statement. \\
Anumana & Logical inference. Three types: Purvavat (cause→effect), Sheshavat (effect→cause), Samanyatodrishta (general correlation). \\
Upamana & Knowledge through comparison or analogy to known solved cases. Used for case-based reasoning and structural similarity mapping. \\
Shabda & Authoritative testimony or established logical principles. Includes universal logical rules, mathematical axioms, and established principles. \\
Pancha Avayava & Five-member syllogism, the core deductive structure. Each syllogism contains five required components. \\
Pratijna & Thesis statement in syllogism; the claim being established. Must be specific and testable. \\
Hetu & Reason or logical ground supporting the thesis. Must reference Pramanas from the evidence sources phase. \\
Udaharana & Universal example with invariable concomitance. Must contain "Wherever X, there is Y" structure (Vyapti) plus a concrete instance (Drishtanta). \\
Vyapti & Invariable concomitance; universal rule stating "wherever X, there is Y." Required component of Udaharana. \\
Drishtanta & Concrete example demonstrating the universal rule. Provided alongside Vyapti in Udaharana. \\
Upanaya & Application of the universal rule to the specific case at hand. Maps the general principle to the particular problem. \\
Nigamana & Conclusion drawn from the syllogism. Restates the thesis, now justified through the preceding four components. \\
Tarka & Counterfactual reasoning; reductio ad absurdum. Verifies conclusions by assuming the opposite and deriving contradictions. \\
Hetvabhasa & Fallacies in reasoning; pseudo-reasons that appear valid but contain logical errors. Five types must be checked. \\
Savyabhichara & Erratic/irregular reasoning. Reason correlates with conclusion but doesn't cause it (correlation vs. causation errors). \\
Viruddha & Contradictory reasoning. Reason actually proves the opposite of the conclusion (logical contradictions). \\
Prakaranasama & Contextually inappropriate reasoning. Circular reasoning or off-topic arguments (begging the question). \\
Sadhyasama & Question-begging reasoning. Premise needs as much proof as the conclusion (assuming what needs to be proved). \\
Kalaatita & Temporally invalid reasoning. Reasoning depends on invalid temporal assumptions (using outdated information as if current). \\
Nirnaya & Ascertainment; definitive knowledge. Distinguishes established knowledge (Prama) from hypothesis requiring verification. \\
Vada & Proper philosophical debate for collaborative truth-seeking. \\
Jalpa & Sophisticated debate aimed at victory through valid argumentation. \\
Vitanda & Critical debate focused on finding flaws without proposing alternatives. \\
\bottomrule
\end{tabular}
\end{table}

\subsection{Data Format Specification}
\label{app:data_format}

\subsubsection{Markdown Structure Template}

Every training example follows a structured markdown format with YAML frontmatter for machine-readable metadata. The complete structure is shown below:

\begin{verbatim}
---
id: pramana-[stage]-[number]
problem_type: constraint_satisfaction | boolean_sat | multi_step_deduction
difficulty: simple | moderate | complex
variables: [number]
ground_truth: "[Expected answer]"
metadata:
  created_date: YYYY-MM-DD
  author: manual | synthetic
  validated: true | false
  z3_verifiable: true | false
  stage: 0 | 1 | 2
---

# Problem

[Natural language problem statement]

**Constraints**:
1. [Constraint 1]
2. [Constraint 2]
...

**Question**: [What needs to be determined]

---

## Samshaya (Doubt Analysis)

**Doubt Type**: [One of 5 categories]

**Justification**: [Why this doubt exists]

---

## Pramana (Evidence Sources)

### Pratyaksha (Direct Perception)
```yaml
observable_facts:
  - "Fact 1 (verbatim or paraphrase)"
  - "Fact 2"
```

### Anumana (Inference)
```yaml
inferences:
  - type: purvavat | sheshavat | samanyatodrishta
    premise: "Starting fact"
    conclusion: "Derived fact"
    justification: "Logical connection"
```

### Upamana (Comparison)
```yaml
analogies:
  - reference: "Similar case"
    similarity: "Structural mapping"
```

### Shabda (Authoritative Principles)
```yaml
principles:
  - "Universal logical rule"
```

---

## Pancha Avayava (Systematic Reasoning)

### Syllogism 1: [Topic]

**Pratijna (Thesis)**: [Claim]

**Hetu (Reason)**: [Evidence]

**Udaharana (Universal + Example)**: Wherever [general rule], 
there [consequence]. For example, [concrete instance].

**Upanaya (Application)**: [How rule applies here]

**Nigamana (Conclusion)**: Therefore, [thesis restated]

[Repeat for each reasoning step]

---

## Tarka (Counterfactual Testing)

**Hypothesis**: Assume [opposite of conclusion].

[Derivation of contradiction]

Therefore, [original conclusion must be true].

---

## Hetvabhasa (Fallacy Detection)

```yaml
fallacy_checks:
  savyabhichara: none_detected | [description]
  viruddha: none_detected | [description]
  prakaranasama: none_detected | [description]
  sadhyasama: none_detected | [description]
  kalaatita: none_detected | [description]

reasoning: "[Why no fallacies detected OR corrections made]"
```

---

## Nirnaya (Definitive Conclusion)

**Status**: Definitive Knowledge | Hypothesis Requiring Verification

**Answer**: [Final answer]

**Justification**: [Why certain OR what evidence missing]

**Confidence**: [High/Medium/Low with explanation]
\end{verbatim}

\subsubsection{Validation Schema Requirements}

Programmatic validation checks the following requirements:

\begin{itemize}
    \item \textbf{YAML Frontmatter}: Must contain \texttt{id}, \texttt{problem\_type}, and \texttt{ground\_truth} fields
    \item \textbf{Required Sections}: All six phases (Samshaya, Pramana, Pancha Avayava, Tarka, Hetvabhasa, Nirnaya) must be present
    \item \textbf{Pramana Completeness}: All four Pramana types (Pratyaksha, Anumana, Upamana, Shabda) must be present
    \item \textbf{Pancha Avayava Structure}: Each syllogism must contain all five components (Pratijna, Hetu, Udaharana, Upanaya, Nigamana)
    \item \textbf{Udaharana Universal Rule}: Each Udaharana must contain "Wherever X, there is Y" structure
    \item \textbf{Hetvabhasa Checks}: All five fallacy types must be explicitly checked
\end{itemize}

\subsubsection{Example File Structure}

A complete example (abbreviated) demonstrating the format is provided in Section~\ref{app:sample_outputs}.

\subsection{Training Hyperparameters}
\label{app:hyperparams}

\subsubsection{Stage 0 Hyperparameters}

Table~\ref{tab:hyperparams_stage0} shows the complete hyperparameter configuration for Stage 0 training.

\begin{table}[h]
\centering
\caption{Stage 0 training hyperparameters.}
\label{tab:hyperparams_stage0}
\begin{tabular}{ll}
\toprule
Parameter & Value \\
\midrule
Base Model & Llama 3.2-3B-Instruct \\
Quantization & 4-bit (QLoRA) \\
LoRA Rank & 64 \\
LoRA Alpha & 64 \\
Target Modules & All attention + FFN \\
Learning Rate & 2e-5 \\
Epochs & 30 \\
Batch Size & 2 \\
Gradient Accumulation & 4 \\
Effective Batch Size & 8 \\
Max Sequence Length & 4096 \\
Warmup Steps & 4 \\
Weight Decay & 0.01 \\
Scheduler & cosine \\
Optimizer & adamw\_8bit \\
Precision & bf16 \\
Hardware & Single A100 (40GB) \\
Training Time & ~8 hours \\
\bottomrule
\end{tabular}
\end{table}

\subsubsection{Stage 1 Hyperparameters}

Table~\ref{tab:hyperparams_stage1} shows the complete hyperparameter configuration for Stage 1 training.

\begin{table}[h]
\centering
\caption{Stage 1 training hyperparameters.}
\label{tab:hyperparams_stage1}
\begin{tabular}{ll}
\toprule
Parameter & Value \\
\midrule
Base Model & DeepSeek-R1-Distill-Llama-8B \\
Quantization & 4-bit (QLoRA) \\
LoRA Rank & 64 \\
LoRA Alpha & 64 \\
Target Modules & All attention + FFN \\
Learning Rate & 2e-5 \\
Epochs & 10 \\
Batch Size & 1 \\
Gradient Accumulation & 4 \\
Effective Batch Size & 4 \\
Max Sequence Length & 4096 \\
Warmup Steps & 4 \\
Weight Decay & 0.01 \\
Scheduler & cosine \\
Optimizer & adamw\_8bit \\
Precision & bf16 \\
Hardware & Single A100 (40GB) \\
Training Time & ~20 hours \\
\bottomrule
\end{tabular}
\end{table}

\subsection{Sample Model Outputs}
\label{app:sample_outputs}

This section provides representative examples of model-generated reasoning traces, demonstrating both successful outputs and common failure modes.

\subsubsection{Complete Successful Output (Stage 1)}

Example~\ref{ex:stage1_success} shows a complete, correctly formatted output from Stage 1 model on test-001 (Alice/Bob/Carol pet assignment problem).

\begin{example}[Complete Stage 1 Output]
\label{ex:stage1_success}
\begin{verbatim}
## Samshaya (Doubt Analysis)
**Doubt Type**: Vipratipatti (Conflicting possibilities to determine)
**Justification**: We need to determine which person has which pet 
based on the given constraints.

---

## Pramana (Sources of Knowledge)
### Pratyaksha (Direct Perception)
- Alice does not have the dog.
- Bob has the cat.
- Carol does not have the fish.

### Anumana (Inference)
- Since Bob has the cat, the remaining pets (dog and fish) must 
  be distributed between Alice and Carol.
- Alice cannot have the dog, so she must have the fish.
- Carol cannot have the fish, so she must have the dog.

### Upamana (Comparison)
- This is a standard assignment problem where each person receives 
  one unique item.

### Shabda (Testimony)
- If a person cannot have an item, they must have one of the 
  remaining items.

---

## Pancha Avayava (5-Member Syllogism)
### Syllogism 1: Assigning Pets to People
**Pratijna (Thesis)**: Alice has the fish, Bob has the cat, and 
Carol has the dog.
**Hetu (Reason)**: Alice cannot have the dog, Bob has the cat, and 
Carol cannot have the fish.
**Udaharana (Universal + Example)**: Wherever a person cannot have 
an item, they must have one of the remaining items.
**Upanaya (Application)**: Alice cannot have the dog, so she must 
have the fish; Carol cannot have the fish, so she must have the dog.
**Nigamana (Conclusion)**: Therefore, Alice has the fish, Bob has 
the cat, and Carol has the dog.

---

## Tarka (Counterfactual Reasoning)
**Hypothesis**: Suppose Alice does not have the fish or Carol does 
not have the dog.
**Consequence**: This would contradict the constraints (Alice cannot 
have the dog, Carol cannot have the fish).
**Analysis**: The constraints fix the assignments uniquely.
**Resolution**: Therefore, Alice has the fish, Bob has the cat, and 
Carol has the dog.

---

## Hetvabhasa (Fallacy Check)
Check for Savyabhichara: No
Check for Viruddha: No
Check for Asiddha: No
Check for Satpratipaksha: No
Check for Badhita: No

---

## Nirnaya (Ascertainment)
**Final Answer**: Alice has the fish, Bob has the cat, and Carol 
has the dog.
**Justification**: The constraints uniquely determine the assignments.
**Confidence**: High
\end{verbatim}
\end{example}

\subsubsection{Partial Output with Missing Sections (Stage 0)}

Example~\ref{ex:stage0_partial} shows a Stage 0 output that is truncated, missing the Hetvabhasa and Nirnaya sections. This demonstrates a common failure mode where the model generates correct reasoning but fails to complete all required phases.

\begin{example}[Partial Stage 0 Output]
\label{ex:stage0_partial}
\begin{verbatim}
## Samshaya (Doubt Analysis)
**Doubt Type**: Samshaya (Doubt about the truth values of P, Q, and R)
**Justification**: We are given three boolean statements P, Q, and R, 
and four facts about their relationships. However, we are not given 
any direct information about the truth values of P, Q, and R. We must 
use the given facts to infer their truth values.

---

## Pramana (Sources of Knowledge)
### Pratyaksha (Direct Perception)
- None

### Anumana (Inference)
- From fact 3, we know Q is true.
- From fact 1, if P is true, then Q is true. Since Q is true, P must 
  also be true (by modus ponens).
- From fact 4, we know P is false. This contradicts our conclusion 
  from fact 1. Therefore, fact 1 must be false, and P is false.
- From fact 2, if Q is true, then R is true. Since Q is true, R must 
  also be true (by modus ponens).

### Upamana (Comparison)
- No direct comparison is possible.

### Shabda (Testimony)
- None

---

## Pancha Avayava (5-Member Syllogism)
### Syllogism 1: 
**Pratijna (Thesis)**: The truth values of P, Q, and R are P = false, 
Q = true, and R = true.
**Hetu (Reason)**: From fact 1: If P is true, then Q is true. Since P 
is false, B cannot be true. (Valid: modus tollens)
**Udaharana (Universal + Example)**: If A is true, then B is true; 
if A is false, then B is false.
**Upanaya (Application)**: If A is true, then B is true. If A is false, 
then B is false. Since A is false, B must be false.
**Nigamana (Conclusion)**: Therefore, A and B are both false.

---

## Tarka (Counterfactual Reasoning)
**Hypothesis**: Suppose P is true.
**Consequence**: If P is true, then Q is true (by fact 1). If Q is true, 
then R is true (by fact 2). Therefore, if P is true, then R is true.
**Analysis**: This is a valid syllogism. The conclusion follows logically 
from the premises.
**Resolution**: Therefore, if P is true, then R is true.

[Output truncated - missing Hetvabhasa and Nirnaya sections]
\end{verbatim}
\end{example}

\subsubsection{Ground Truth Example}

Example~\ref{ex:ground_truth} shows a complete ground truth reasoning trace from the training data (pramana-001), demonstrating the expected structure and depth of reasoning.

\begin{example}[Ground Truth Training Example]
\label{ex:ground_truth}
\begin{verbatim}
## Samshaya (Doubt Analysis)
**Doubt Type**: Samana Dharma Upapatti (Multiple possibilities share 
similar properties)
**Justification**: There are three people and three pets, creating 
multiple possible assignments. Without systematic reasoning, we cannot 
determine which person has which pet.

---

## Pramana (Sources of Knowledge)
### Pratyaksha (Direct Perception)
- "Alice does not have the cat"
- "Bob has the dog"
- "Carol does not have the fish"
- "There are exactly three people: Alice, Bob, Carol"
- "There are exactly three pets: cat, dog, fish"
- "Each person has exactly one pet"
- "Each pet belongs to exactly one person"

### Anumana (Inference)
- type: purvavat
  premise: "Bob has the dog (directly stated)"
  conclusion: "Neither Alice nor Carol has the dog"
  justification: "Since each pet belongs to exactly one person, if Bob 
  has the dog, no one else can have it"

---

## Pancha Avayava (5-Member Syllogism)
### Syllogism 1: Establishing Bob's Pet
**Pratijna (Thesis)**: Bob has the dog.
**Hetu (Reason)**: This is directly stated in constraint 2.
**Udaharana (Universal + Example)**: Wherever a constraint directly 
assigns a pet to a person, that assignment is true. For example, if 
we are told "X has Y," then X has Y.
**Upanaya (Application)**: In this specific problem, constraint 2 
states "Bob has the dog." This is a direct assignment, so the universal 
rule applies: Bob has the dog.
**Nigamana (Conclusion)**: Therefore, Bob has the dog.

---

## Tarka (Counterfactual Reasoning)
**Hypothesis**: Suppose Carol does not have the cat (negation of our 
conclusion).
**Consequence**: If Carol does not have the cat, then Carol must have 
either the dog or the fish. However, Bob has the dog (established), 
so Carol cannot have the dog. Carol also cannot have the fish 
(constraint 3). Therefore, Carol would have no pet, which violates 
the completeness principle.
**Analysis**: This leads to an absurdity: Carol would have no pet, yet 
the problem states each person has exactly one pet.
**Resolution**: Therefore, our original conclusion must be true. Carol 
must have the cat.

---

## Hetvabhasa (Fallacy Check)
fallacy_checks:
  savyabhichara: none_detected
  viruddha: none_detected
  prakaranasama: none_detected
  sadhyasama: none_detected
  kalaatita: none_detected

---

## Nirnaya (Ascertainment)
**Status**: Definitive Knowledge
**Final Answer**: Alice has the fish, Bob has the dog, and Carol has 
the cat.
**Justification**: All constraints are satisfied. The reasoning follows 
valid logical principles, all possibilities have been systematically 
eliminated, and Tarka testing confirms the solution.
**Confidence**: High
\end{verbatim}
\end{example}

\subsection{Evaluation Details}
\label{app:evaluation}

\subsubsection{Stage 0 Per-Example Results}

Table~\ref{tab:stage0_per_example} shows detailed per-example evaluation results for Stage 0 model on the 10 test examples.

\begin{table}[h]
\centering
\caption{Stage 0 per-example evaluation results.}
\label{tab:stage0_per_example}
\small
\begin{tabular}{lccc}
\toprule
Example ID & Parse Success & Semantic Match & Error Type \\
\midrule
pramana-003 & Yes & No & -- \\
pramana-005 & No & -- & Missing Hetvabhasa \\
test-001 & No & -- & Missing Pancha Avayava \\
test-002 & No & -- & Missing Tarka Analysis field \\
test-003 & Yes & Yes & -- \\
test-004 & No & -- & Missing Hetvabhasa \\
test-005 & Yes & Yes & -- \\
test-006 & No & -- & Missing Nirnaya Justification \\
test-007 & No & -- & Invalid Pancha Avayava structure \\
test-008 & Yes & No & -- \\
\bottomrule
\end{tabular}
\end{table}

\subsubsection{Stage 1 Per-Example Results}

Table~\ref{tab:stage1_per_example} shows detailed per-example evaluation results for Stage 1 model on the 10 test examples.

\begin{table}[h]
\centering
\caption{Stage 1 per-example evaluation results.}
\label{tab:stage1_per_example}
\small
\begin{tabular}{lccc}
\toprule
Example ID & Parse Success & Semantic Match & Error Type \\
\midrule
pramana-003 & No & -- & Missing Nirnaya Justification \\
pramana-005 & No & -- & Missing Hetvabhasa \\
test-001 & Yes & Yes & -- \\
test-002 & No & -- & Invalid doubt type \\
test-003 & No & -- & Invalid doubt type \\
test-004 & No & -- & Missing Nirnaya \\
test-005 & No & -- & Missing Hetvabhasa \\
test-006 & Yes & Yes & -- \\
test-007 & Yes & Yes & -- \\
test-008 & Yes & Yes & -- \\
\bottomrule
\end{tabular}
\end{table}

\subsubsection{Parse Error Categorization}

Table~\ref{tab:parse_errors_appendix} breaks down parse errors by category for both stages.

\begin{table}[h]
\centering
\caption{Parse error categorization breakdown (detailed).}
\label{tab:parse_errors_appendix}
\begin{tabular}{lcc}
\toprule
Error Type & Stage 0 Count & Stage 1 Count \\
\midrule
Missing Hetvabhasa section & 2 & 3 \\
Missing Nirnaya section & 1 & 1 \\
Missing Nirnaya Justification field & 1 & 1 \\
Missing Pancha Avayava section & 1 & 0 \\
Invalid Pancha Avayava structure & 1 & 0 \\
Missing Tarka Analysis field & 1 & 0 \\
Invalid doubt type & 0 & 2 \\
\bottomrule
\end{tabular}
\end{table}

\subsubsection{Summary Statistics}

Table~\ref{tab:eval_summary} provides summary statistics comparing Stage 0 and Stage 1 performance.

\begin{table}[h]
\centering
\caption{Evaluation summary statistics.}
\label{tab:eval_summary}
\begin{tabular}{lcc}
\toprule
Metric & Stage 0 & Stage 1 \\
\midrule
Format Adherence Rate & 40\% & 40\% \\
Semantic Match Rate & 50\% & 100\% \\
Parse Success Rate & 40\% & 40\% \\
Average Output Length (tokens) & 3,200 & 2,900 \\
Complete 6-Phase Outputs & 4/10 & 4/10 \\
\bottomrule
\end{tabular}
\end{table}
